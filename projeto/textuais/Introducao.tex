\chapter{Introdução}\label{CAP:introducao}

\noindent Este documento consiste de um modelo básico para a produção de documentos academicos, seguindo as normas ABNT. 

Não e abordado o estudo do LaTex neste template. Sugerimos a leitura do texto em \citeonline{Oetiker:1995}. O uso do LaTex e aconselhavel devido a sua qualidade grafica, facil referenciacao,
 criacao de listas, indices, referencias bibliograficas e escrita matematica profissional. Porem, nao e obrigatorio o uso deste template, apenas as orientacoes de formatação segundo as normas ABNT devem ser obrigatoriamente seguidas.

Uma observação em particular é a de que, no pacote ABNTex, as referências diretas devem utilizar o comando ``citeonline''. Referências indiretas utilizam o comando ``cite''.

Exemplo de citacao direta: Uma otima fonte de estudo para compreender o LaTex e apresentada por \citeonline{Oetiker:1995}. 

Exemplo de citação indireta: Existem boas fontes de pesquisa para entendimento do LaTex~\cite{Oetiker:1995,radke:2005}, estas incluem documentação online disponível na web.

Uma citação da Internet~\cite{cil:2016,phdthesis:2011}.

Uma dissertação de mestrado~\cite{mastersthesis:93} ou~\citeonline{mastersthesis:93} de acordo com~\citeonline{Oetiker:1995}.

Uma tese de doutorado~\cite{phdthesis:2011} ou~\citeonline{phdthesis:2011}.

\section{Motivação e objetivos}


 
\section{Contribuicoes}




\section{Organizacao do trabalho}

\noindent \textbf{Capitulo \ref{trab_rela}}: descricao...

\noindent \textbf{Capitulo \ref{meto}}: descricaoo...

\noindent \textbf{Capitulo \ref{desen}}: descricao...

\noindent \textbf{Capitulo \ref{result}}: descricao...