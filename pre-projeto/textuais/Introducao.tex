\chapter{Redações e aprendizado de máquina.}\label{sec:intro}

O desenvolvimento de uma redação e uma atividade prática presente na cultura civilizada desde a invenção da escrita.
Já faz pelo menos uma década que um bom desempenho na redação do Exame Nacional de Ensino Médio - ENEM  virou sinônimo de chances maiores para ser aprovado no processo seletivo de acesso a inúmeras universidades públicas ~\cite{sisu:2017} e a importantes programas de governo ~\cite{csf:2017}.

Em todo processo seletivo é comum o uso de marcações em gabaritos afim de automatizar o processo de correção, uma alternativa rápida e segura, até mesmo aplicações de provas eletrônicas são cada vez mais comum. Um exemplo seria o processo seletivo para avaliadore das redações do ENEM que durante a ``FASE II'' respondem uma prova eletrônica eliminatória ~\cite{paq_a:2016}. É notável que todo o processo evoluiu com objetivo de agilidade, confiança e segurança do resultado, entretanto a avaliação das competências de uma redação ainda depende exclusivamente da supervisão de duas ou mais pessoas envolvidas ~\cite{edital_enem:2016}.

Machine Learning ou aprendizado de máquina é uma área de IA cujo objetivo é o desenvolvimento de técnicas computacionais sobre o aprendizado bem como a construção de sistemas capazes de adquirir conhecimento de forma automática. Os algoritmos de aprendizado de máquina procuram padrões dentro de um conjunto de dados~\cite{machine_learning:1997}. Esses algoritmos existem há bastante tempo, uma ciência que não é nova, mas que está ganhando um novo impulso enquanto o processamento computacional cresce e fica mais barato.

A hipótese desta monografia é que a aplicação do aprendizado de máquina na valoração das competências de uma redação pode ter sua eficiência e eficácia aprimoradas.

\section{Problema de Pesquisa}

A redação é aplicada no ENEM desde a primeira edição 1998, hoje o maior exame do Brasil, que na edição de 2016 teve 8.627.195 escritos confirmados, e a participação direta de 11.360 profissionais externos na correção de 5.825.134 redações, entre eles, 378 supervisores e 10.982 avaliadores de acordo com a CEBRASPE ~\cite{relatorio_de_gestao:2016}. 

Segundo o edital do ENEM 2016 ~\cite{edital_enem:2016} cada redação foi avaliada por, pelo menos, dois avaliadores, de forma independente, contabilizando um número mínimo de 11.650.268 avaliações manuais, das competências exigidas em um texto de redação pelo ENEM.

Dado um corpus de redações avaliar as competencias de uma nova redação substituindo a etapa de avaliação manual.

\section{Motivação}

Com crescente volume e variedade de dados disponíveis, o processamento computacional que está mais barato e mais poderoso, e o armazenamento de dados de forma acessível, o aprendizado de máquina está no centro de muitos avanços tecnológicos atingindo áreas antes exclusivas de seres humanos. Os carros autônomos do Google são o exemplo de uma atividade antes exclusiva de um humano e hoje exercida e aperfeiçoada por algoritmos de aprendizado de máquina ~\cite{waymo:2017}.

Aplicações de aprendizado de máquina estão presentes na nossa vida cotidiana como, resultados de pesquisa web, análise de sentimento baseado em texto e na detecção de fraudes em operações com cartões de crédito ~\cite{batista1999aplicando}, um dos usos mais óbvios e importantes em nosso mundo de hoje.

As competências exigidas em uma redação podem ser avaliadas com aprendizado de máquina. Uma aplicação de aprendizado de máquina está livre de ansiedade, fadiga, \textit{stress} entre outro fatores que afetão a tomada de decisão. Isto representaria a análise e classificação de um texto imparcial a opinião do autor classificando todas as competências necessárias para definição de uma nota.

\subsection{Objetivos Específicos}

Dentro do escopo geral de forma detalhada e refinada as ações que se prentende executar para alcançar o objetivo principal acima, são particularizadas como os seguintes objetivos específicos:

\begin{itemize}
 \item Montar um corpus de redações com textos em prosa, do tipo dissertativo-argumentativo com temas diversificados, onde todas as competências foram avaliadas seguindo os mesmo criterios do ENEM.
 \item Montar um segundo corpus de redações com temas diversificados, onde todas as competências não foram avaliadas. Este segundo corpus será necessário para definição do próximo item.
 \item Entre os algoritmos de aprendizado supervisionado e semi-supervisionado definir qual atende o escopo geral do problema de forma simples sem perda de qualidade dos resultados.
 \item Do algoritimo acima definido encontrar os modelos capazes de avaliar as competências do texto de uma redação e treina-los.
 \item Realizar os seguintes teste sobre os modelos treinados.
  \begin{itemize}
   \item Acurácia (taxa de erro): a taxa de predições corretas (ou incorretas) realizada pelo modelo para um determinado conjunto de dados.
   \item \textit{Overfitting} (super-ajustamento): ocorre quando o modelo se especializa nos dados utilizados no seu treinamento, apresentando uma taxa de acurácia baixa para novos dados.
   \item \textit{Noise} (ruído): classificação errada do conjunto de dados de entrada.
  \end{itemize}
 \item Comparar os resultados obtidos entre os testes de cada modelo utilizado na avaliação das competências do texto.
 \item Embasado nos resultados da comparação, montar uma aplicação web, capaz de receber um texto em prosa, do tipo dissertativo-argumentativo, sobre um tema de ordem social e avaliar suas competências para atribuir uma nota. 
\end{itemize}

\newpage
\section{Metodologia}

Para concluir com êxito os objetivos propostos o método utilizado para solução do problema tem os seguintes passo:

\begin{itemize}
\item Linguagens, Frameworks e bibliotecas
 \begin{itemize}
  \item Definição e justificativa da linguagem utilizada.
  \item Definição e justificativa das bibliotecas utilizadas.
  \item Definição e justificativa das frameworks utilizadas.
 \end{itemize}
\item Coleta de dados
 \begin{itemize}
  \item A partir de fontes de dados disponíveis na web, coletar e organizar um corpus de redações com textos em prosa, do tipo dissertativo-argumentativo de temas variados, avaliados segundo as competências exigidas pelo ENEM.
  \item Dividir 75\% do corpus para ser utilizado na etapa de treinamento do modelo e os demais 25\% restantes para testes posteriores.
 \end{itemize}
\item Algoritmo e modelo
 \begin{itemize}
  \item Avaliar a viabilidade do uso de um algoritimo supervisionado ou semi-supervisionado.
  \item Definição dos modelos que atendem o escopo geral do problema de avaliação de redações.
 \end{itemize}
\item Treinamento e teste.
 \begin{itemize}
  \item Treinamento dos modelos avaliados.
  \item Teste de acurácia, \textit{overfitting} e \textit{noise} do modelos avaliados.
 \end{itemize}
\item Resultados.
 \begin{itemize}
  \item Comparar gráficamente o resultado dos testes de cada modelo avaliado entre cada um.
  \item Elaboração da conclusão dos teste e viabilidade de uso de cada modelo.
 \end{itemize}
\item Aplicação prática.
 \begin{itemize}
  \item Embasado na conclusão dos testes utilizar os modelos em uma aplicação web. 
 \end{itemize}
\end{itemize}

\newpage
\section{Levantamento bibliográfico}

Os principais trabalhos recentes em avaliação de redações:

\begin{itemize}
 \item \citeonline{joao_nobre:2011}
\end{itemize}
