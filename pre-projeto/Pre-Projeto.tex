\documentclass[ruledheader,noindentfirst,anapcustomindent,abntfigtabnum,tocpage=plain]{abnt}

\usepackage{amsmath, amssymb, amsthm, verbatim, amsfonts, amstext}
%\usepackage[latin1]{inputenc}
\usepackage[brazilian]{babel}
\usepackage[utf8]{inputenc}
\usepackage[T1]{fontenc}
\usepackage{dropping}
\usepackage{graphicx}
\usepackage[hang,small,bf]{caption}
\usepackage[abnt-etal-list=0,abnt-etal-text=it,abnt-and-type=&,abnt-emphasize=bf,abnt-full-initials=yes,alf,bibjustif]{abntcite}
\usepackage{fancyhdr}
\usepackage{makeidx}
\usepackage[none]{hyphenat}
\usepackage{color}
\usepackage{subfig}
\usepackage{algorithms}
\usepackage{algorithmic}
\usepackage{mdwlist}
\usepackage{bm}
\usepackage[titletoc,title]{appendix}
\usepackage{ltxtable}
\usepackage{longtable}
\usepackage{supertabular}
\usepackage{indentfirst}
\usepackage{color}
\usepackage{icomma}
\usepackage{url}

\usepackage{float}
\usepackage{multirow}
\usepackage{longtable}
\usepackage{enumerate}
\sloppy

%
%Tradução do pacote Algorithm para portugues
%
\renewcommand{\algorithmicrequire}{\textbf{Entrada:}}
\renewcommand{\algorithmicensure}{\textbf{Saída:}}
\renewcommand{\algorithmicend}{\textbf{fim}}
\renewcommand{\algorithmicif}{\textbf{se}}
\renewcommand{\algorithmicthen}{\textbf{então}}
\renewcommand{\algorithmicelse}{\textbf{senão}}
\renewcommand{\algorithmicelsif}{\algorithmicelse \, \algorithmicif}
\renewcommand{\algorithmicendif}{\algorithmicend \, \algorithmicif}
\renewcommand{\algorithmicfor}{\textbf{para}}
\renewcommand{\algorithmicforall}{\textbf{para todo}}
\renewcommand{\algorithmicdo}{\textbf{fazer}}
\renewcommand{\algorithmicendfor}{\algorithmicend \, \algorithmicfor}
\renewcommand{\algorithmicwhile}{\textbf{enquanto}}
\renewcommand{\algorithmicendwhile}{\algorithmicend \, \algorithmicwhile}
\renewcommand{\algorithmicloop}{\textbf{laço}}
\renewcommand{\algorithmicendloop}{\algorithmicend \, \algorithmicloop}
\renewcommand{\algorithmicrepeat}{\textbf{repetir}}
\renewcommand{\algorithmicuntil}{\textbf{até}}
\renewcommand{\algorithmiccomment}[1]{\{#1\}}
\renewcommand{\listalgorithmname}{Lista de Algoritmos}
\floatname{algorithm}{Algoritmo}
%%%%%%%%%%%%%%%%%%%%%%%%%%%%%%%%%%%%%%%%%%%%%%%%%%%%%%%%%%%%%%%%%%%%%%%%%%%%%%%%%%%
\newcommand*\rot{\rotatebox{90}}
\newcommand*\V{\ding{51}}
\newcommand\tab[1][1cm]{\hspace*{#1}}

\makeindex

%%%% O arquivo estiloCAP.tex possui as definições para ciação do estilo de capítulo (fonte de título, barras horizontais, etc.)
% ele não gera texto de saída, é um arquivo de configuração somente
%
% \input{configuracao/estiloCAP}
%%%%%%%%%%%%%%%%%%%%%%%%%%%%%%%%%%%%%%%%%%%%%%%FIM DO PREAMBULO%%%%%%%%%%%%%%%%%%%%%%%%%%%%%%%%%%%%%%%%%%%%%%%%%%%%%%%%%%%%%%%%%%




\begin{document}

%%%%% IMPORTANTE: ALTERA O TEXTO ENTRE ARIAL E TIMES NEW ROMAN (ALTERNAR OS COMENTÁRIOS)
%
%%%%%%%%%%%%%%%%%%%%%PARA UTILIZAR ARIAL%%%%%%%%%%%%%%%%%%%%%%%
%
\fontfamily{phv}                    %fonte Arial
\renewcommand{\rmdefault}{phv}      %
%
%%%%%%%%%%%%%%%%%%%%%PARA UTILIZAR TIMES%%%%%%%%%%%%%%%%%%%%%%%
%
%\fontfamily{ptm}               %fonte Times
%\renewcommand{\rmdefault}{ptm} %
%
%%%%%%%%%%%%%%%%%%%%%%%%%%%%%%%%%%%%%%%%%%%%%%%%%%%%%%%%%%%%%%%

%%%%%%%%%%%%%Arquivos .tex com os elementos pré-textuais
%
\thispagestyle{empty}

\vfill
 \begin{center}
    \begin{figure}[t]
     \centering
            % \includegraphics[width=5cm]{figures/logo.pdf}\\[-0.1in]
            \includegraphics[width=15cm]{figuras/logo-uit.pdf}\\[-0.1in]
     \end{figure}

    {\large\bfseries UNIVERSIDADE DE ITAÚNA} \\
    {\large\bfseries PRÓ-REITORIA DE ENSINO} \\
    {\large\bfseries COORDENAÇÃO DE CIÊNCIA DA COMPUTACAO}  \\ 
    {\large\bfseries BACHARELADO EM CIÊNCIA DA COMPUTAÇÃO}  \\ 

    \vspace*{1in}
    \begin{large} \bfseries Eugênio Cunha\end{large}\\[0.4in]

    \vspace*{4cm}
    \noindent \\
    \large\bfseries{APRENDIZADO DE MÁQUINA APLICADO À VALORAÇÃO DAS COMPETÊNCIAS DE UMA REDAÇÃO} \\
    \vfill
    \large\bfseries{ ITAÚNA \\ 2017}
\end{center}

\normalsize

%%%%%%%%%%%%%%%%%%%%%%%%%%%%%%%%%%%%%%%%%%%%%%%%%%%%%%%%%%%%%%%%%%%%

%Capítulos passam a ter páginas numeradas
%
\pagestyle{fancy}

%resseta os contadores de capítulo e seção
%
\renewcommand{\chaptermark}[1]{\markboth{#1}{}}
\renewcommand{\sectionmark}[1]{\markright{\thesection\ #1}}



%%% Outros arquivos .tex. É acoselhável utilizar vários arquivos, pelo menos um por capítulo
\chapter{Introdução}\label{CAP:introducao}

\noindent O desenvolvimento de uma redação e uma atividade prática presente na cultura civilizada desde a invenção da escrita.
Já faz pelo menos uma década que um bom desempenho na redação do Exame Nacional de Ensino Médio - ENEM  virou sinônimo de chances maiores para ser aprovado no processo seletivo de acesso a inúmeras universidades públicas ~\cite{sisu:2017} e a importantes programas de governo como Ciência sem fronteiras ~\cite{csf:2017}.

Em todo processo seletivo é comum o uso de marcações em gabaritos afim de automatizar o processo de correção, uma alternativa rápida e segura, até mesmo aplicações de provas eletrônicas são cada vez mais comum. Um exemplo seria o processo seletivo para avaliador das redações do ENEM que durante a ``FASE II'' respondem uma prova eletrônica eliminatória ~\cite{paq_a:2016}. É notável que todo o processo evoluiu com objetivo de agilidade, confiança e segurança do resultado, entretanto a avaliação das competências de uma redação ainda depende exclusivamente da supervisão de duas ou mais pessoas envolvidas ~\cite{edital_enem:2016}.

A redação é aplicada no ENEM desde a primeira edição 1998, hoje o maior exame do Brasil, que na edição de 2016 teve 8.627.195 escritos confirmados, e a participação direta de 11.360 profissionais externos na correção de 5.825.134 redações, entre eles, 378 supervisores e 10.982 avaliadores de acordo com a CEBRASPE ~\cite{relatorio_de_gestao:2016}. 

Segundo o edital do ENEM 2016 ~\cite{edital_enem:2016} cada redação foi avaliada por, pelo menos, dois avaliadores, de forma independente, contabilizando um número mínimo de 11.650.268 avaliações manuais, das competências exigidas em um texto de redação pelo ENEM.

\textit{Machine Learning} ou aprendizado de máquina é uma área de IA cujo objetivo é o desenvolvimento de técnicas computacionais sobre o aprendizado bem como a construção de sistemas capazes de adquirir conhecimento de forma automática. Os algoritmos de aprendizado de máquina procuram padrões dentro de um conjunto de dados~\cite{machine_learning:1997}. Esses algoritmos existem há bastante tempo, uma ciência que não é nova, mas que está ganhando um novo impulso enquanto o processamento computacional cresce e fica mais barato.

A hipótese desta monografia é que um ou mais modelos de aprendizado de máquina na valoração das competências de uma redação pode ser tão eficiênte quanto o processo de avaliação manual.

\section{Definição do Problema de Pesquisa}

Dado um corpus de redações avaliar as competências exigidas em um texto de redação do tipo dissertativo-argumentativo substituindo a etapa de avaliação manual.

\section{Motivação}

Com crescente volume e variedade de dados disponíveis, o processamento computacional que está mais barato e mais poderoso, e o armazenamento de dados de forma acessível, o aprendizado de máquina está no centro de muitos avanços tecnológicos atingindo áreas antes exclusivas de seres humanos. Os carros autônomos do Google são o exemplo de uma atividade antes exclusiva de um humano e hoje exercida e aperfeiçoada por algoritmos de aprendizado de máquina ~\cite{waymo:2017}.

Aplicações de aprendizado de máquina estão presentes na nossa vida cotidiana como, resultados de pesquisa web, análise de sentimento baseado em texto e na detecção de fraudes em operações com cartões de crédito ~\cite{batista1999aplicando}.

As competências exigidas em uma redação podem ser avaliadas por aprendizado de máquina, diferente de um ser humano um algoritmo de aprendizado de máquina está livre de ansiedade, fadiga, \textit{stress} entre outro fatores emocionais que afetão uma avaliação imparcial a opinião do autor.

\section{Objetivos Gerais e Específicos}

Este trabalho tem como objetivo geral aplicar aprendizado de máquina na avaliação das competências exigidas em um texto de redação do tipo dissertativo-argumentativo.

\subsection{Objetivos Especificos}

O método de construção do conhecimento deste trabalho terá como fundamentos processos de pesquisas relacionadas às áreas descritas. O mesmo será dividido em etapas dentro do escopo geral de forma detalhada e refinada para alcançar o objetivo geral acima, são particularizadas como os seguintes objetivos específicos:

\begin{itemize}
 \item Percorrer o banco de redações UOL ~\cite{uol_banco_redacoes:2017} em páginas HTML, filtrar e coletar redações avaliadas;
 \item Normalizar os textos coletados, separar o tema, título, texto e competências avaliadas em uma estrura no formato JSON;
 \item Montar um fluxo de trabalho utilizando a ferramenta para mineração de dados \textit{Orange} ~\cite{JMLR:demsar13a} com modelos classificadores de multiplas classes;
 \item Ajustar e treinar os modelos classificadores com o corpus de redações; 
 \item Realizar testes de acurácia, \textit{overfitting} e \textit{noise} sobre os classificadores;
 \item Representar e comparar graficamente os resultados obtidos. 
\end{itemize}

\section{Contribuições}

O presente estudo contribuirá na área do aprendizado de máquina e diretamente no processo de avaliação de um texto em prosa do tipo dissertativo argumentativo.

\section{Organização do trabalho}

\noindent \textbf{Capitulo \ref{trab_rela}}: Trabalhos Relacionados cita alguns dos trabalhos lidos para  embasamento teórico que serviram de base para solucionar o problema proposto.

\noindent \textbf{Capitulo \ref{meto}}: Método proposto apresenta as etapas passo a passo para desenvolver e resolver o problema proposto deste trabalho.

\noindent \textbf{Capitulo \ref{desen}}: Desenvolvimento descreve cada procedimento metodológico que será
utilizado para a realização da pesquisa.

\noindent \textbf{Capitulo \ref{result}}: Resultados Experimentais apresenta os resultados obtidos do trabalho desta pesquisa.
\chapter{Plano de Trabalho}\label{plano}

continua...

% O candidato se dispõe ao regime de tempo-integral, com dedicação de 40 horas semanais para o projeto e não possui nenhum vínculo empregatício.

% As seguintes disciplinas já foram cursadas:

% \vspace{0.2cm}
% \begin{tabular}{r m{8cm} l}
%     $\bullet$ & Disciplina 1 & ENE \\
%     $\bullet$ & Disciplina 2 & CIC \\
% \end{tabular}
% \vspace{0.4cm}

% A sigla ENE representa as disciplinas do departamento de Engenharia Elétrica e CIC as disciplinas do departamento de Ciência da Computação.

% As disciplinas presencias do doutorado serão cursadas no período X. As disciplinas a serem cursadas são:

% \vspace{0.2cm}
% \begin{tabular}{r m{8cm} l}
%     $\bullet$ & Disciplina 3 & ENE \\
%     $\bullet$ & Disciplina 4 & CIC \\
% \end{tabular}
% \vspace{0.4cm}

% \newpage
% \section{Cronograma}

% O cronograma de atividades é dado pela tabela abaixo:

% \begin{center}
% \rowcolors{1}{white}{gray!25}
% \begin{tabular}{c l |c|c|c|c|c|c|c|c|}
%     & Atividade & \rot{2012-1} & \rot{2012-2} & \rot{2013-1} & \rot{2013-2} & \rot{2014-1} & \rot{2014-2} & \rot{2015-1} & \rot{2015-2} \\
%     \hline
%     1 & Cursar Disciplinas      &\V &\V &\V &   &   &   &   &   \\
%     2 & Estágio de Docência     &   &   &   &\V &\V &   &   &   \\
%     3 & Revisão Bibliográfica   &\V &\V &\V &\V &\V &\V &\V &   \\
%     4 & Elaboração do Algoritmo &   &   &\V &\V &\V &\V &\V &   \\
%     5 & Coleta de Dados         &   &   &   &\V &\V &\V &   &   \\
%     6 & Análise dos Resultados  &   &   &   &\V &\V &\V &\V &   \\
%     7 & Preparação de artigos   &   &   &   &   &\V &\V &\V &\V \\
%     8 & Escrita da Monografia   &   &   &   &   &   &\V &\V &\V \\
%     9 & Entrega da Monografia   &   &   &   &   &   &   &   &\V \\
%     \hline
% \end{tabular}
% \end{center}

    



%%%% Estilo de citação ABNT e arquivo de bibitens (bib.bib)
\bibliographystyle{configuracao/abnt-alf}
\bibliography{bibliografia/bib}


\end{document}